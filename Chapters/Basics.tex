\chapter{Basics}

\section{Wikidata}

Every \textit{item} on Wikidata has a \textit{label} by which it can be identified and a unique item number that looks something like this \textit{Q12345}. Throughout this thesis I will use the example of the item Douglas Adams (Q42). Every \textit{item} can also have diambiguitions or alternative names that it might be known under. As for example in Adam's case this is Douglas Noele Adams, including his middle name. A description can be added to every item, as in Adam's case this is 'English writer and humorist'. Every \textit{item} has a list of \textit{statements} about that item. For a person this could be gender, parents, first and last name, date of birth, sex, siblings, occupation, children and so forth. These are called \textit{properties} which have \textit{values} and potentially a \textit{qualifier}. So for example the \textit{statement} with the \textit{property} last name has Adams as a \textit{value}. If we look at the spouse \textit{property} we see that the \textit{value} is \textit{Jane Belson}. This \textit{value} has \textit{qualifiers} to specify when exactly this \textit{statement} was true. In the case of Adams' spouse it is a start time and an end time. Every \textit{property} can have multiple \textit{values}. For example one can have multiple spouses or multiple children. These don't have to be mutually exclusive. This whole package is then called a \textit{claim}. Every \textit{value} of a \textit{statement} should preferably have a \textit{reference} to back the \textit{claim} that was made. 

Other noteworthy things are that an \textit{item} links all articles of all languages together. The \textit{items} are not bound by languages. All referenced pictures and all other references to one of the Wikimedia projects are also interlinked at Wikidata. 

Like Wikipedia, Wikidata can be edited anonymously or with a Wikimedia account. 

\clearpage
\section{Editing on the Wikipedia}

Unless otherwise specified, I will be talking about editing articles with the VisualEditor throughout this thesis. So first it is important to mention, that the VisualEditor is not available by default on all the Wikipedias. It is available as a gadget feature on all Wikipedias though, but only the english Wikipedia has it activated by default. The VisualEditor is an editing tool that lets you edit via a GUI rather than constructing the whole article in wikitext. When editing an infobox the expected statements are already provided, rather than having to look them up and write them down manually. 

\section{Non-Functional Requirements}

